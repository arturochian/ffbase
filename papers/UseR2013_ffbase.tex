\documentclass[11pt, a4paper]{article}
\usepackage{amsfonts, amsmath, hanging, hyperref, natbib, parskip, times}
\usepackage[pdftex]{graphicx}
\hypersetup{
  colorlinks,
  linkcolor=blue,
  urlcolor=blue
}

\let\section=\subsubsection
\newcommand{\pkg}[1]{{\normalfont\fontseries{b}\selectfont #1}} 
\let\proglang=\textit
\let\code=\texttt 
\renewcommand{\title}[1]{\begin{center}{\bf \LARGE #1}\end{center}}
\newcommand{\affiliations}{\footnotesize}
\newcommand{\keywords}{\paragraph{Keywords:}}

\setlength{\topmargin}{-15mm}
\setlength{\oddsidemargin}{-2mm}
\setlength{\textwidth}{165mm}
\setlength{\textheight}{250mm}

\begin{document}
\pagestyle{empty}

\title{\pkg{ffbase}: statistical functions for large datasets}

\begin{center}
  {\bf Edwin de Jonge$^{1}$, Jan Wijffels$^{2}$}
\end{center}

\begin{affiliations}
1. Statistics Netherlands,  \href{mailto:e.dejonge@cbs.nl}{e.dejonge@cbs.nl} \\[-2pt]
2. BNOSAC - Belgium Network of Open Source Analytical Consultants,  \href{mailto:jwijffels@bnosac.be}{jwijffels@bnosac.be} \\[-2pt]
\end{affiliations}

\keywords Large datasets, memory constraints, modelling on large data

\vskip 0.8cm
Statistical datasets used to be small, but nowadays it is not uncommon that the dataset is too large to be handled in R without encountering the frequently encountered \textbf{Error: cannot allocate vector of size \ldots Mb} issue.\\

To handle the \proglang{R} memory constraints, the \pkg{ff} package (Adler \& Oehlschl\"{a}gel et al.) was developed in 2008. It handles the memory constraint by storing data on disk.
For day-to-day data munging, frequently used functionality from the \pkg{base} package had to be developed to make it more easy for an \proglang{R} developer to work with package \pkg{ff}. For this, the \pkg{ffbase} package has been developed to extend the \pkg{ff} package to allow basic statistical operations on large data frames, especially \emph{ffdf} objects.\\

The \pkg{ffbase} package contains a lot of the functionality from the R's base package for usage with large datasets through package \pkg{ff}. 
Namely 
\begin{itemize}
  \item Basic operations (c, unique, duplicated, ffmatch, ffdfmatch, \%in\%, is.na, all, any, cut, ffwhich, ffappend, ffdfappend, rbind, ffifelse, ffseq, ffrep.int, ffseq\_len)
  \item Standard operators $(+, -, *, /, \hat{}, \%\%, \%/\%, ==, !=, <, \le, \ge, >, \&, |, !)$ working on ff vectors
  \item Math operators (abs, sign, sqrt, ceiling, floor, trunc, round, signif, log, log10, log2, log1p, exp, expm1, acos, acosh, asin, asinh, atan, atanh, cos, cosh, sin, sinh, tan, tanh, gamma, lgamma, digamma, trigamma)
  \item Selections \& data manipulations (subset, transform, with, within, ffwhich)
  \item Summary statistics (sum, min, max, range, quantile, hist, binned\_sum, binned\_tabulate)
  \item Data transformations (cumsum, cumprod, cummin, cummax, table.ff, tabulate.ff, merge, ffdfdply, as.Date, format)
  \item Chunked functionalities (chunkify), writing \& loading data (load.ffdf, save.ffdf, move.ffdf, laf\_to\_ffdf)
\end{itemize}
For modelling purposes, \pkg{ffbase} has bigglm.ffdf to allow to build generalized linear models easily on large data and can connect to the \pkg{stream} package for clustering \& classification.

In the presentation, the \pkg{ffbase} package will be showcased to show that working with large datasets without having RAM issues in \proglang{R} is easy and natural for an \proglang{R} programmer.

%% references: 
\bibliographystyle{chicago}

\subsubsection*{References}
Daniel Adler, Christian Gläser, Oleg Nenadic, Jens Oehlschl\"{a}gel and
  Walter Zucchini (2013). ff: memory-efficient storage of large data on
  disk and fast access functions. R package version 2.2-11.
  \url{http://CRAN.R-project.org/package=ff}
  
Edwin de Jonge, Jan Wijffels and Jan van der Laan (2011). ffbase:
  Basic statistical functions for package ff. R package version 0.7-1.
  \url{http://github.com/edwindj/ffbase}


\end{document}
